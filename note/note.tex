\documentclass[a4j,twocolumn]{jarticle}
\usepackage[dvipdfmx]{graphicx}
\usepackage{graphicx}
\usepackage{amsmath}
\usepackage{txfonts}
\usepackage{listings}
\usepackage{fancyhdr}
\begin{document}
  \pagestyle{fancy}
  \lhead{AIwolf 2018}
  \chead{Agent Sol}
  \rhead{\thepage}
  \cfoot{}
\section{このエージェントについて}
  プロトコル部門向けのエージェントになります。
  エージェント本体はPython3で記述しています。
  パッケージはnumpy, scipy, pandasなどを使用しています。
  通信などについては公開されているHaradaさんのAIwolfPyのものを使用しています。
\section{起動方法}
  \begin{enumerate}
    \item AIwolfを人狼知能HPよりDL
    \item 適当な場所に展開
    \item StartServer.shを実行、人数を決め、Connectをクリック
    \item python3 Sol.py -h localhost -p 10000を実行
  \end{enumerate}
  人数合わせのために他のjavaエージェントなどをつなぐ場合はStartGUIClient.shを実行するとすぐにできる
\section{メソッド}
  \subsection{talk}
    turn制で、生存プレイヤが全員一斉に発話する。
    returnで話す内容を指定する。
    プロトコル部門ではcontentbuilderのメソッドを返し、自然言語部門では文字列を返す。
  \subsection{whisper}
    人狼が2人以上生存している場合にのみ実行可能。
    人狼にしか送信されないtalk。
    returnで話す内容を指定する。
    プロトコル部門ではcontentbuilderのメソッドを返し、自然言語部門では文字列を返す。
  \subsection{vote}
    処刑対象の指定。
    returnで投票対象のエージェント番号を指定する。
  \subsection{attack}
    人狼のみ可能なメソッド。
    returnで攻撃先のエージェント番号を指定する。
  \subsection{divine}
    占い師のみ可能なメソッド。
    returnで占い先のエージェント番号を指定する。
  \subsection{guard}
    狩人のみ可能なメソッド。
    returnで護衛先のエージェント番号を指定する。
  \subsection{initialize}
    ゲームの初期化時に呼ばれるメソッド。
  \subsection{update}
    更新情報がある時に呼ばれるメソッド。
    Java版と挙動が異なるため注意。
    daily\_finishやrequestを内包する。
  \subsection{dayStart}
    前日のいろいろな結果が来るメソッド。
  \subsection{finish}
    ゲーム終了時に呼ばれるメソッド。
    何故か2回呼ばれるらしい。
\section{データ構造}
  \subsection{base\_info}
    辞書型で基礎的な情報すべてを保持している。
    initializeやupdate時に渡される。
    \begin{itemize}
      \item[agentIdx] 自身のエージェント番号
      \item[myRole] 自身のエージェントの役職
      \item[roleMap] 役職のdict。他のエージェントの役職が確定的にわかる場合(人狼の場合など)はここに入る
      \item[statusMap] エージェントの生死を表すdict
      \item[remainTalkMap] 生存エージェントのその日の残り発話可能回数を表すdict
      \item[remainWhisperMap] Agentのその日の残りの可能な囁き回数を表すdict
    \end{itemize}
  \subsection{diff\_data}
    pandas DataFrameで、ゲームに関わる情報を保持している。
    6列存在し、{agent, day, idx, text, turn, type}がある。
    typeによって内容の意味が異なる。
    \subsubsection{type = initialize, finish}
      agent = idx = agentIdx (自分自身) \par
      initializeの場合はday = 0 \par
      turn = 0 \par
      text = comingout文 \par
    \subsubsection{type = talk, whisper}
      agent = 発話者 \par
      day = 日付 \par
      idx = talk/whisperのid \par
      turn = talk/whisperのturn \par
      text = 発話文そのまま
    \subsubsection{type = attack\_vote, vote}
      agent = 投票対象 \par
      idx = 投票者 \par
      turn = 0 再投票時は-1 \par
      text = vote/attack文
    \subsubsection{type = execute, dead}
      agent = 死者 \par
      idx = turn = 0 \par
      text = Over
    \subsubsection{type = divine, identify, guard}
      agent = 能力の対象 \par
      idx = 能力使用者 \par
      turn = 0 \par
      text = devined/identified/guarded文
    \subsubsection{type = attack}
      agent = 襲撃対象 \par
      idx = turn = 0 \par
      text = attack文
\end{document}